% !TEX program = xelatex
\documentclass[usenames,dvipsnames,handout,aspectratio=169]{beamer}

\usefonttheme{professionalfonts}
\usepackage[no-math]{fontspec}
\usepackage{multicol}

\usepackage[none]{hyphenat}
\usepackage{hyperref}

\fontspec [Path = ./fonts/, 
	UprightFont = TextbookNew-Light,
	BoldFont = TextbookNew-Regular]
{Textbook New}

\defaultfontfeatures{Scale=MatchLowercase}
\setmainfont[Path=fonts/,
	UprightFont = TextbookNew-Light,
	BoldFont = TextbookNew-Regular,
%	ItalicFont = TextbookNew-LightItalic,
%	BoldItalicFont = TextbookNew-Italic
]{Textbook New}
\setromanfont[Path=fonts/,
	UprightFont = TextbookNew-Light,
	BoldFont = TextbookNew-Regular
%	ItalicFont = TextbookNew-LightItalic,
%	BoldItalicFont = TextbookNew-Italic
]{Textbook New}
\setsansfont[Path=fonts/,
	UprightFont = TextbookNew-Light,
	BoldFont = TextbookNew-Regular
%	ItalicFont = TextbookNew-LightItalic,
%	BoldItalicFont = TextbookNew-Italic
]{Textbook New}
\setmonofont[Path=fonts/, UprightFont = InputMono-Regular]{Input Mono}
\newfontfamily{\cyrillicfont}[ Path=fonts/ ]{TextbookNew-Light}
\newfontfamily{\cyrillicfontbf}[ Path=fonts/ ]{TextbookNew-Regular}
\newfontfamily{\cyrillicfonttt}[Path=fonts/, UprightFont = InputMono-Regular]{Input Mono}

\usepackage{yandex}
\usepackage{amsmath,amssymb,amsfonts,amsthm}
\usepackage{mathtools}

\setdefaultlanguage{english} % язык определяет перевод названий, в том числе и логотипа

\title{Sequence Modelling}
\subtitle{Fifth Machine Learning in High Energy Physics Summer School,\\
MLHEP 2019}

\author[Ekaterina Artemova]{
  Ekaterina Artemova
}
\institute{
	\begin{tabular}{c c}
		National~Research~University~Higher~School~of~Economics
	\end{tabular}
}

\onbehalf{RNN, biRNN, BPTT, LSTM, RecNN}

\noyandexlogoframe


%%%%%%%%%%%%%%%%%%%%%%%%%%%%%
\usepackage{graphicx}
\DeclareGraphicsExtensions{.eps,.png}
\graphicspath{ {img/} }

% \usefonttheme[onlymath]{serif}

% added by Evgeny Sokolov
\def\XX{\mathbb{X}}
\def\PP{\mathbb{P}}
\def\FF{\mathcal{F}}
\def\EE{\mathbb{E}}
\def\NN{\mathcal{N}}
\def\LL{\mathcal{L}}
\def\YY{\mathbb{Y}}
\def\RR{\mathbb{R}}
\def\HH{\mathbb{H}}
\def\AA{\mathcal{A}}
\newcommand{\cond}{\mspace{3mu}{|}\mspace{3mu}}
\newcommand{\argmin}{\mathop{\rm arg\,min}\limits}

\usepackage{tikz}
\usetikzlibrary{positioning,arrows,shapes.geometric,shadows,trees}
\usetikzlibrary{calc}

\tikzstyle{format} = [thick, draw=red!50!black!50]
\usepackage{tkz-graph}

\usepackage{color}
\usepackage{bm}
\usepackage{animate}

\setlength{\abovedisplayskip}{2mm}
\setlength{\belowdisplayskip}{2mm}
%%%%%%%%%%%%%%%%%%%%

\begin{document}

% Несколько альтернатив основного логотипа на титульную страницу:
%\yandexlogo                          % общий логотип Яндекса
%\yandexservicelogo{Название сервиса} % логотип конкретного сервиса

%\yandexschoollogo                    % логотип ШАД
% \crayfislogo
%\yandexdatafactorylogo               % логотип YDF

% Дополнительные логотипы. Здесь можно указывать произвольный код. \textheight обозначает допустимую высоту.
\addlogo{\includegraphics[height=\textheight,keepaspectratio]{yandex/hse.eps}}
\addlogo{\includegraphics[height=\textheight,keepaspectratio]{yandex/logo_shad.eps}}
%\addlogo{\includegraphics[height=\textheight,keepaspectratio]{yandex/logo_skoltech.png}}
% \addlogo{\includegraphics[height=\textheight,keepaspectratio]{yandex/mipt.eps}}
% \addlogo{\includegraphics[height=\textheight,keepaspectratio]{yandex/st.png}}
% \addlogo{\includegraphics[height=\textheight,keepaspectratio]{yandex/ucd_seal.eps}}
% \addlogo{\includegraphics[height=\textheight,keepaspectratio]{yandex/uci.eps}}
% %\addlogo{\insertyandexdataschoollogo[height=\textheight,keepaspectratio]}
% %\addlogo{\insertyandexdatafactorylogo[height=\textheight,keepaspectratio]}


\begin{frame}[plain]
\titlepage
\end{frame}

\begin{frame}[plain]
\tableofcontents
\end{frame}

\section{Sequence modelling}

\begin{frame}
\frametitle{Sequential data}

\begin{enumerate}
	\item Time series
	\begin{itemize}
	\item Financial data analysis: stock market, commodities, Forex
	\item Healthcare: pulse rate, sugar level (from medical equipment and wearables) 
	\end{itemize}
	\item Text and speech: speech understanding, text generation
	\item Spatiotemporal data
	\begin{itemize}
	\item Self-driving and object tracking
	\item Plate tectonic activity
	\end{itemize}
	\item Physics: jet identification
	\item etc.
\end{enumerate}

\end{frame}




\begin{frame}
\frametitle{Sequence modelling I}

\begin{block}{Sequence classification}

\begin{enumerate}


\item $\bm{x} = x_1,x_2,\ldots,x_n $, $x_i \in V$  - objects
\item $y \in \{1,\ldots, L\}$ - labels 
\item $ \{(\bm{x}^{(1)}, y_1 ), (\bm{x}^{(2)}, y_2 ), \ldots, (\bm{x}^{(m)}, y_m ) \} $ – training data

\end{enumerate}

Classification problem: $ \gamma : \bm{x} \rightarrow y$ 
	
\end{block}

\begin{enumerate}
	\item Activity recognition: $x$ – pulse rate, $y$ – activity (walking, running, peace)
	\item Opinion mining: $x$ – sentence, $y$ – sentiment (positive, negative)
	\item Trading: $x$ – stock market, $y$ – action (sell, buy, do nothing)

\end{enumerate}

\end{frame}


\begin{frame}
\frametitle{Sequence modelling II}

\begin{block}{Sequence labelling}

\begin{enumerate}


\item $\bm{x} = x_1,x_2,\ldots,x_n $, $x_i \in V$  - objects
\item $\bm{y} = y_1,y_2,\ldots,y_n $, $y_i \in \{1,\ldots, L\}$ - labels 
\item $ \{(\bm{x}^{(1)}, \bm{y}^{(1)} ), (\bm{x}^{(2)}, \bm{y}^{(2)} ),\ldots, (\bm{x}^{(m)}, \bm{y}^{(m)} ) \} $ – training data
\item exponential number of possible solutions : if $\texttt{length}(x) =n$, there are  $L^n$ possible solutions
\end{enumerate}

Classification problem: $ \gamma : \bm{x} \rightarrow \bm{y}$ 
	
\end{block}

\begin{enumerate}

	\item Part of speech tagging: $x$ – word, $y$ – part of speech (verb, noun, etc.)
	\item Genome annotation: $x$ – DNA, $y$ – genes
	\item HEP tracking: $x$ - a set of hits with backgrounds, $y$ – hit classification


\end{enumerate}

\end{frame}

\begin{frame}
\frametitle{Sequence modelling III}

\begin{block}{Sequence transduction / transformation}

\begin{enumerate}


\item $\bm{x} = x_1,x_2,\ldots,x_n $, $x_i \in V_{source}$  - objects
\item $\bm{y} = y_1,y_2,\ldots,y_n $, $y_i \in V_{target}$ - objects 
\item $ \{(\bm{x}^{(1)}, \bm{y}^{(1)} ), (\bm{x}^{(2)}, \bm{y}^{(2)} ),\ldots, (\bm{x}^{(m)}, \bm{y}^{(m)} ) \} $ – training data
\item $\bm{x}^{(1)}$, $\bm{y}^{(1)} $ are of different length

\end{enumerate}

Transduction problem: $\bm{x}_{source} \rightarrow \bm{y}_{target}$ 
	
\end{block}

\begin{enumerate}

	\item Machine translation: $x$ – sentence in German, $y$ – sentence in English
	\item Speech recognition: $x$ – spoken language, $y$ – text
	\item Chat bots: $x$ – question, $y$ – answer


\end{enumerate}

\end{frame}


\begin{frame}{Traditional ML approaches to sequence modeling}
\begin{columns}
\begin{column}{0.6\textwidth}

\begin{itemize}
	
\item Hidden Markov Models (HMM)
\item Conditional Random Fields (CRF)
\item Local classifier: for each $x$ define features, based on $x_{-1}$, $x_{+1}$, etc,  and perform classification $n$ times
\end{itemize}

Problems:
\begin{enumerate}
	\item Markov assumption: fixed length history
	\item Computation complexity
 
\end{enumerate}
\end{column}
\begin{column}{0.4\textwidth}  %%<--- here
    \begin{center}
	\includegraphics[width=1.\textwidth]{../../img/HMM.png}
     \end{center}
\end{column}
\end{columns}
\end{frame}


\begin{frame}{DL approaches to sequence modeling}
\begin{columns}
\begin{column}{0.6\textwidth}

\begin{itemize}
	
\item Recurrent neural network and its modifications: LSTM, GRU, Highway
\item Transformer
\item 2D Convolutional Neural Network
\item Pointer network
\end{itemize}

Problems:
\begin{enumerate}
	\item Training time
	\item Amount of training data
\end{enumerate}
\end{column}
\begin{column}{0.4\textwidth}  %%<--- here
    \begin{center}
	\includegraphics[width=1.\textwidth]{../img/seqmodel.png}
     \end{center}
\end{column}
\end{columns}
\end{frame}

\section{Recurrent neural network}
\subsection{Definition}

\begin{frame}{Recurrent neural network}	
\begin{columns}
\begin{column}{0.6\textwidth}

\begin{itemize}
	
\item  Input: sequence of vectors 
\item 
$x_{1:n} = x_1, x_2, \ldots, x_n$, $x_i \in \mathbb{R}^{d_{in}}$

\item  Output: a single vector

 $y_n = RNN(x_{1:n})$, $y_n \in \mathbb{R}^{d_{out}}$

\item For each prefix $x_{i:j}$ define an output vector $y_i$:

$y_i = RNN(x_{1:i})$

\item $RNN^{*}$ is a function returning this sequence for input sequence $x_{1:n}$:

$y_{1:n} = RNN^{*}(x_{1:n})$, $y_i \in \mathbb{R}^{d_{out}}$

\end{itemize}

\end{column}
\begin{column}{0.4\textwidth}  %%<--- here
    \begin{center}
	\includegraphics[width=1.\textwidth]{../../img/rnn1.png}
     \end{center}
\end{column}
\end{columns}
\end{frame}


\begin{frame}{Sequence modelling with RNN}

\begin{enumerate}

	\item Sequence classification  

Put a dense layer on top of RNN to predict the desired class of the sequence after the whole sequence is processed 
\[ p( l_j | \bm{x}_{1:n} ) = \texttt{softmax} (RNN(\bm{x}_{1:n}) \times W + b)_{[j]} \]


	\item Sequence labelling 
	
Produce an output $y_i$ for each input RNN reads in. Put a dense layer on top of each output to predict the desired class of the input

\[ p( l_j | \bm{x}_{j}  ) = \texttt{softmax} (RNN(\bm{x}_{1:j}) \times W + b)_{[j]} \]


\end{enumerate}
	
\end{frame}


\begin{frame}{More details on RNN}	
\begin{columns}
\begin{column}{0.6\textwidth}

\begin{itemize}
	
\item $RNN^{*}(x_{1:n}, s_0) = y_{1:n}$ 
\item $y_i = O(s_i)$ – simple activation function 
\item $s_i = R(s_{i-1, x_i})$, where $R$ is a recursive function, $s_i$ is a state vector
\item $s_0$ is initialized randomly or is a zero vector
\item $x_i \in \mathbb{R}^{d_{in}}$, $y_i \in \mathbb{R}^{d_{out}}$, $s_i \in \mathbb{R}^{f(d_{out})}$ 
\item $\theta$ –  shared weights

\end{itemize}

\end{column}
\begin{column}{0.4\textwidth}  %%<--- here
    \begin{center}
	\includegraphics[width=1.\textwidth]{../../img/rnn1.png}
     \end{center}
\end{column}
\end{columns}
\end{frame}




\begin{frame}{More details on RNN}	
\begin{columns}
\begin{column}{0.6\textwidth}

\begin{itemize}
	
\item $s_i = R(x_i, s_{i-1}) =g(s_{i-1}  W^s + x_i W^x + b)$
\item $y_i = O(s_i) = s_i$
\item $y_i, s_i, b \in \mathbb{R}^{d_{out}}$, $x_i \in \mathbb{R}^{d_{in}}$
\item $W^x \in \mathbb{R}^{d_{in} \times d_{out}}$, $W^s \in \mathbb{R}^{d_{out} \times d_{out}}$

\end{itemize}

\end{column}
\begin{column}{0.4\textwidth}  %%<--- here
    \begin{center}
	\includegraphics[width=1.\textwidth]{../../img/rnn1.png}
     \end{center}
\end{column}
\end{columns}
\end{frame}



\begin{frame}{RNN unrolled}	

    \begin{center}
	\includegraphics[width=0.7\textwidth]{../../img/rnn2.png}
     \end{center}

\[ s_4 = R(s_3, x_4) = R(R(s_2, x_3), x_4) = R(R(R(s_1, x_2), x_3), x_4) =  \]
\[ = R(R(R(R(s_0, x1), x_2), x_3), x_4)\]

\end{frame}

\begin{frame}{Bidirectional RNN (Bi-RNN)}

The input sequence can be read from left to right and from right to left. Which direction is better?	

    \begin{center}
	\includegraphics[width=0.7\textwidth]{../../img/birnn1.png}
     \end{center}

\[biRNN(x_{1:n}, i) = y_i = [RNN^f(x_{1:i}); RNN^r(x_{n:i})]\]
\end{frame}


\begin{frame}{Bi-RNN}	

    \begin{center}
	\includegraphics[width=0.7\textwidth]{../../img/birnn2.png}
     \end{center}

\[biRNN^{*}(x_{1:n}, i) = y_{1:n} = biRNN(x_{1:n}, 1) \ldots biRNN(x_{1:n}, n)\]
\end{frame}

\begin{frame}{Multilayer RNN}	

    \begin{center}
	\includegraphics[width=0.7\textwidth]{../../img/rnn8.png}
     \end{center}


Connections between different layers are possible too: $y_1^2 = \texttt{concat}(x_1, y_1^1)$


\end{frame}

\subsection{Training}





\begin{frame}{Sequence classification}	

\begin{columns}
\begin{column}{0.5\textwidth}


\begin{itemize}
\item $\hat{y_n} = O(s_n)$
\item $\texttt{prediction} = MLP(\hat{y_n})$
\item Loss: $L(\hat{y_n}, y_n)$
\item $L$ can take any  form: cross entropy, hinge, margin, etc.
\end{itemize}


\end{column}

\begin{column}{0.5\textwidth}

\includegraphics[width=1.0\textwidth]{../../img/rnn3.png}
	
\end{column}
\end{columns}
\end{frame}


\begin{frame}{Sequence labelling}	


\begin{columns}
\begin{column}{0.5\textwidth}


\begin{itemize}
	\item Output $\hat{t_i}$ for each input $x_{1,i}$
	\item Local loss: $L_{local}( \hat{t_i}, t_i)$
	\item Global loss: $L(\hat{t_n}, t_n ) = \sum_i L_{local}(\hat{t_i}, t_i)$
	\item $L$ can take any  form: cross entropy, hinge, margin, etc.
\end{itemize}

\end{column}
\begin{column}{0.5\textwidth}


\includegraphics[width=1.0\textwidth]{../../img/sq1.png}
\end{column}
\end{columns}
\end{frame}



\begin{frame}{Backpropogation through time}	
\begin{center}
\includegraphics[width=0.7\textwidth]{../../img/bptt.png}

\end{center}





$s_i = R(x_i, s_{i-1}) =g(s_{i-1}  W^s + x_i W^x + b)$

Chain rule: $ \frac{\partial L}{\partial w} = \frac{\partial L}{\partial p(\hat{y_5})} \frac{\partial p(\hat{y_5})}{\partial s_4} (\frac{ \partial s_4 }{\partial w}  + \frac{ \partial s_4 }{\partial s_3} \frac{ \partial s_3 }{\partial w} + \frac{ \partial s_4 }{\partial s_3} \frac{ \partial s_3 }{\partial s_2} \frac{ \partial s_2 }{\partial s_w} + \ldots ) $




\end{frame}

\begin{frame}{Vanishing gradient problem}	


Chain rule: $ \frac{\partial L}{\partial w} = \frac{\partial L}{\partial p(\hat{y_5})} \frac{\partial p(\hat{y_5})}{\partial s_4} (\frac{ \partial s_4 }{\partial w}  + \frac{ \partial s_4 }{\partial s_3} \frac{ \partial s_3 }{\partial w} + \frac{ \partial s_4 }{\partial s_3} \frac{ \partial s_3 }{\partial s_2} \frac{ \partial s_2 }{\partial s_w} + \ldots ) $

$g$ – sigmoid

\begin{enumerate}
	\item Many sigmoids near 0 and 1
	\begin{itemize}
		\item Gradients $\rightarrow$ 0
		\item Not training for long term dependencies
	\end{itemize}
	\item Many sigmoids $>1$
	\begin{itemize}
		\item Gradients $\rightarrow +\inf$ 
		\item Not training again
	\end{itemize}
\end{enumerate}
Solution: gated architectures (LSTM and GRU)

\end{frame}

\section{Gated architectures}

\begin{frame}{Controlled memory access}
\begin{itemize}
	\item Entire memory vector is changed: $s_{i+1} = R(x_i, s_i)$
	\item Controlled memory access: $s_{i+1} = g \odot R(x_i, s_i) + (1-g) s_i$
	
	$g \in [0,1]^d,  s,x \in \mathbb{R}^d$
	\item Differential gates: $\sigma(g), g' \in \mathbb{R}^d$
	\item This controllable gating mechanism is the basis of the LSTM and the GRU architectures
	
\end{itemize}	
\end{frame}

\begin{frame}{Long short term memory}
\begin{center}
\includegraphics[width=1.0\textwidth]{../../img/lstm1.png}
\end{center}

\url{http://colah.github.io/posts/2015-08-Understanding-LSTMs/}	
\end{frame}


\begin{frame}{Long short term memory}
\begin{center}
\includegraphics[width=1.0\textwidth]{../../img/lstm2.png}
\end{center}

\url{http://colah.github.io/posts/2015-08-Understanding-LSTMs/}	
\end{frame}

\begin{frame}{Long short term memory}
\begin{center}
\includegraphics[width=1.0\textwidth]{../../img/lstm3.png}
\end{center}

\url{http://colah.github.io/posts/2015-08-Understanding-LSTMs/}	
\end{frame}


\begin{frame}{Long short term memory}
\begin{center}
\includegraphics[width=1.0\textwidth]{../../img/lstm4.png}
\end{center}

\url{http://colah.github.io/posts/2015-08-Understanding-LSTMs/}	
\end{frame}

\begin{frame}{Long short term memory}
\begin{center}
\includegraphics[width=1.0\textwidth]{../../img/lstm5.png}
\end{center}

\url{http://colah.github.io/posts/2015-08-Understanding-LSTMs/}	
\end{frame}

\begin{frame}{Gated recurrent unit}
\begin{center}
\includegraphics[width=1.0\textwidth]{../../img/gru.png}
\end{center}

\url{http://colah.github.io/posts/2015-08-Understanding-LSTMs/}	
\end{frame}

\section{RNN generators}

\begin{frame}{Sequence generation}

Teacher forcing: $x := <s> x, y := x </s?$

$x : <s> x_1 x_2 \ldots x_n$

$y : x_1 x_2 \ldots x_n </s>$

\begin{center}
\includegraphics[width=0.7\textwidth]{../../img/gen.png}
\end{center}




\end{frame}


\begin{frame}{Sequence generation}
\begin{itemize}
	\item Examples of generated texts: \url{http://karpathy.github.io/2015/05/21/rnn-effectiveness/}
	\item Examples of generated MIDI music: \url{https://towardsdatascience.com/how-to-generate-music-using-a-lstm-neural-network-in-keras-68786834d4c5}

\end{itemize}	
\end{frame}

\begin{frame}{Pros and cons of RNNs}

\begin{enumerate}
\item Advantages: 
\begin{itemize} 
\item RNNs are popular and successful for variable-length sequences
\item The gating models such as LSTM are suited for long-range error propagation
\end{itemize}

\item Problems: 
\begin{itemize}
\item The sequentiality prohibits parallelization within instances
\item Long-range dependencies still tricky, despite gating 
\end{itemize}

\end{enumerate}

\end{frame}

\section{The Transformer}


\begin{frame}{The Transformer}	
\begin{columns}
\begin{column}{0.6\textwidth}

An alternative architecture to RNN which allows of parallel and faster training

\begin{itemize}
	
\item Several layers of identical modules
\item Each module consists of Multi-Head Attention and Feed Forward layers
\item Input: embeddings. How to get embeddings for numerical input? Apply any dense layer
\item Posituonal embeddings
\end{itemize}

\end{column}
\begin{column}{0.4\textwidth}  %%<--- here
    \begin{center}
	\includegraphics[width=1.\textwidth]{../../img/transformer.png}
     \end{center}
\end{column}
\end{columns}
\end{frame}


\begin{frame}{Multi-head Attention}	
\includegraphics{}

\[ Attention(K, Q, V) = softmax(\frac{QK^{T}}{\sqrt{d_k)) V\]

\end{frame}

\begin{frame}{Positional embeddings}
... are used to handle sequences, 	
\includegraphics{}


\end{frame}




\begin{frame}{Sequence 2 sequence learning}
Encoder-decoder model for:
\begin{enumerate}
	\item Machine translation
	\item Chit-chat bots
\end{enumerate}
\includegraphics[width=1.0\textwidth]{../../img/seq2seq_1.png}
\end{frame}




\section{Bonus I: seq2seq}

\begin{frame}{Sequence 2 sequence learning}
Encoder-decoder model for:
\begin{enumerate}
	\item Machine translation
	\item Chit-chat bots
\end{enumerate}
\includegraphics[width=1.0\textwidth]{../../img/seq2seq_1.png}
\end{frame}



\section{Bonus II: RecNN}

\begin{frame}{Modeling trees with Recursive NN}
\begin{itemize}
	\item Input: $x_1, x_2, \ldots, x_n$
	\item A binary tree $T$ can be represented as a unique set of triplets~$(i,k,j)$, s.t.~$i < k < j$, $x_{i:j}$ is parent of $x_{i:k}, i_{k+1,j}$
	\item RecNN takes as an input a binary tree and returns as output a corresponding set of inside state vectors $\bm{s_{i:j}^A} \in \mathbb{R}^d$
	\item Each state vector $\bm{s_{i:j}^A}$ represents the corresponding tree node $q_{i:j}^A$ and encodes the entire structure rooted at that node
\end{itemize}
	
\end{frame}


\begin{frame}{RecNN}
\begin{itemize}
\item Input: $x_1, x_2, \ldots, x_n$ and a binary tree $T$

\item $RecNN(x_1, x_2, \ldots, x_n, T) = \{ \bm{s_{i:j}^A} \in \mathbb{R}^d | q_{i:j}^A \in  T \} $
\item $\bm{s_{i:i}^A} = v(x_i)$
\item $\bm{s_{i:j}^A} = R(A,B,C, \bm{s_{i:k}^B}, \bm{s_{k+1:j}^C})$, $q_{i:k}^B \in T, q_{k+1:j}^C \in T$
\item $R(A,B,C, \bm{s_{i:k}^B}, \bm{s_{k+1:j}^C}) = g([\bm{s_{i:k}^B}, \bm{s_{k+1:j}^C}]W)$	
\end{itemize}
\end{frame}

\end{document}
